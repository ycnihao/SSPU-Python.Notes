% !TEX program = xelatex
\documentclass[12pt]{article}
\usepackage{xeCJK}
\usepackage{geometry}
\usepackage{framed}
\usepackage{enumitem}
\usepackage{titlesec}
\usepackage{xcolor}
\usepackage{amsmath}

% 页面设置
\geometry{a4paper, left=25mm, right=15mm, top=20mm, bottom=20mm}

% 设置中文字体(苹果系统)
\setCJKmainfont{PingFang SC}  % 苹方字体
\setCJKsansfont{Hiragino Sans GB}  % 冬青黑体
\setCJKmonofont{STKaiti}  % 华文楷体

% 设置英文字体
\setmainfont{Helvetica Neue}
\setsansfont{Helvetica}
\setmonofont{Menlo}

% 段落格式
\setlength{\parindent}{0em}
\setlength{\parskip}{0.5em}
\linespread{1.3}

% 章节格式
\titleformat{\section}
  {\normalfont\Large\bfseries\sffamily}{\thesection}{1em}{}
\titleformat{\subsection}
  {\normalfont\large\bfseries\sffamily}{\thesubsection}{1em}{}

% 列表格式
\setlist[itemize]{leftmargin=*, topsep=0pt}
\setlist[enumerate]{leftmargin=*, topsep=0pt}

% 错误日志宏:用于在文档中记录简洁的时间戳错误条目
% 用法:\errorentry{YYYY-MM-DD hh:mm}{错误描述}
\newcommand{\errorentry}[2]{\item[\textbf{#1}] #2}

% 定义笔记环境
\newenvironment{note}
  {\begin{framed}\small\itshape}
  {\end{framed}}

\newenvironment{important}
  {\begin{framed}\color{red}\small\bfseries}
  {\end{framed}}

% 代码环境
\usepackage{listings}
\lstset{
basicstyle=\ttfamily\small,
    breaklines=true,
    frame=single,
    numbers=left,
    numberstyle=\tiny\color{gray}
}

% 列表(代码)显示设置:
% - keepspaces: 保留空格(用于代码缩进)
% - showspaces/showstringspaces: 不在输出中显示可见空格符号(例如 ␣)
% 这样可以既保留缩进又不渲染空格可视标记。
\lstset{keepspaces=true, showspaces=false, showstringspaces=false}

\begin{document}

% 标题
\begin{center}
    {\Huge\bfseries SSPU 高级程序语言设计笔记}\\[0.5em]
    {\large 2025年10月11日-2026年1月7日}\\[1em]
\end{center}

% 目录(可选)
\tableofcontents
\newpage


% 正文开始
\section{Python语言概述\textbf{(2025年10月15日)}}

% 这是2025年11月12日的笔记内容。python课程主要学习了以下内容:

% \begin{note}
%   Python中的函数调用是通过函数名加括号来实现的。例如,调用一个名为 \texttt{my\_function} 的函数可以写成 \texttt{my\_function()}。

% \end{note}

% \subsection{子节标题}

% 子节内容可以包含列表:

% \begin{itemize}
%   \item 要点一
%   \item 要点二
%   \item 要点三
% \end{itemize}

% 或者编号列表:

% \begin{enumerate}
%   \item 第一项
%   \item 第二项
%   \item 第三项
% \end{enumerate}

\newpage

\section{基本数据类型\textbf{(2025年10月22日)}}

% 数学公式示例:

% \begin{equation}
%   E = mc^2
% \end{equation}

% 或者行内公式:$a^2 + b^2 = c^2$

% \begin{important}
%   重要信息会用红色边框突出显示。
% \end{important}

\newpage

\section{程序控制结构\textbf{(2025年10月29日)}}

% \begin{lstlisting}[language=Python, caption=Python代码示例]
% def hello_world():
%     print("Hello, World!")
%     return True
% \end{lstlisting}

\newpage

\section{组合数据类型\textbf{(2025年11月5日)}}

\newpage

\section{函数与模块\textbf{(2025年11月12日)}}

\subsection{函数的变量作用域}
  基本概念:\\
    1.局部变量:函数体内声明的\\
    2.全局变量:函数体外声明的\\
  在函数体内定义全局,使用global

\subsection{模块}
import模块 调用为模块.函数名\\
from 模块 import 函数  调用为直接 函数名\\
导入所有函数为 from 模块 import * 调用为直接 函数名\\

\subsection{函数装饰器}
输入为函数,输出也为函数
\subsection{类}
类相当于函数的集合,不过类是一群带有性质的函数的集合
\subsubsection{数据成员 2025年11月26日}
首先可以简要理解类与成员的关系为工厂与产品的关系,类是工厂,成员是产品。\\
类可以通过某种形式来生成对象,就像工厂通过某种生产规则生产出产品一样。\\
\newpage
写一段代码理解一下:
\begin{lstlisting}[language=Python,caption=类的数据成员示例]
class Student:
  school = "SSPU"

  def __init__(self, name, age):
    self.name = name
    self.age = age
  def display_info(self):
    print(f"姓名:{self.name},年龄:{self.age},学校:{self.school}")
student1 = Student("张三",20)
student2 = Student("李四",22)
student1.display_info()  # 输出:姓名:张三, 年龄:20, 学校:SSPU
student2.display_info()  # 输出:姓名:李四, 年龄:22, 学校:SSPU
\end{lstlisting}

\begin{note}
  注意:类变量是所有实例共享的,而实例变量是每个对象独有的。
\end{note}

\begin{lstlisting}[language=Python,caption=私有变量示例]
  class BankAccount:
    def __init__(self,balance):
      self._balance = balance 
    def get_balance(self):
      return self._balance 
    def deposit(self,amount):
      self._balance += amount
  account=BankAccount(1000)
  print(account.get_balance()) #正确方式,可以输出1000
  print(account._balance) #错误方式,私有变量无法访问
\end{lstlisting}
在这里,Student就是一个类,school = "SSPU"是全体实例的属性(也就是都是二工大的学生)

这类似于一个模版,然后具体输入张三和李四,就会生成二工大学生张三以及李四的这样的信息。

而这里的私有变量就相当于学生的隐私,比如张三或者李四的银行账户与存款等等

\newpage
\subsection{基类和派生类}
\begin{lstlisting}[language=Python,caption=基类和派生类示例]
  class Student:    #创建学生基类
    def __init__(self,name,number):
      self.name = name
      self.number = number
    
    def do_homework(self):
      print("学生完成作业")

  class G_student(Student):     #研究生派生类
    def __init__(self,name,number,advisor):
      super().__init__(name,number)
      self.advisor = advisor

    def do_homework(self):
      print("研究生作研究")
    
  class U_student(Student):   #本科派生类
    def __init__(self,name,number,teacher):
      super().__init__(name,number)
      self.teacher = teacher

    def do_homework(self):
      print("本科生找工作")

  def homework(student): #多态函数
    student.do_homework()
  g_student=G_student("张三","2025001","王教授") #研究生张三,学号2025001,导师王教授
  u_student=U_student("李四","2025002","李老师") #本科生李四,学号2025002,授课老师李老师

  homework(g_student) #调用多态函数,输出:研究生学生完成作业 (做研究)
  homework(u_student) #调用多态函数,输出:大学本科生完成作业 (找工作)


\end{lstlisting}
\begin{note}
多态函数的好处在于老师不需要知道每个阶段的学生具体类型,只需要调用homework函数即可,这样的话
不同阶段的学生会做它们自己的任务。
\end{note}
\newpage

\section{我在学习Python过程中出现的错误\textbf{(2025年12月3日)}}
\begin{description}
\errorentry{2025-12-03 12:00}{可见空格符问题:代码块每行前有多余空格,已在导言区添加配置保留缩进但不渲染可见空格。}
\errorentry{2025-12-03 12:02}{构造函数名写错:把 \_init\_ 写成了 \_\_init\_\_(少一个下划线)。}
\errorentry{2025-12-03 12:05}{继承语法错误:类继承写法不正确,应为 class G\_student(Student):。}
\errorentry{2025-12-03 12:06}{中文引号问题:使用了中文引号,应使用标准ASCII双引号。}
\errorentry{2025-12-03 12:08}{代码块嵌套错误:类定义位置不正确。}
\errorentry{2025-12-03 12:10}{注释与输出不一致:建议同步注释与实际输出。}
\end{description}
\end{document}