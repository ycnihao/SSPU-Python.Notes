% !TEX program = xelatex
\documentclass[12pt]{article}
\usepackage{xeCJK}
\usepackage{geometry}
\usepackage{framed}
\usepackage{enumitem}
\usepackage{titlesec}
\usepackage{xcolor}
\usepackage{amsmath}

% 页面设置
\geometry{a4paper, left=25mm, right=15mm, top=20mm, bottom=20mm}

% 设置中文字体(苹果系统)
\setCJKmainfont{PingFang SC}  % 苹方字体
\setCJKsansfont{Hiragino Sans GB}  % 冬青黑体
\setCJKmonofont{STKaiti}  % 华文楷体

% 设置英文字体
\setmainfont{Helvetica Neue}
\setsansfont{Helvetica}
\setmonofont{Menlo}

% 段落格式
\setlength{\parindent}{0em}
\setlength{\parskip}{0.5em}
\linespread{1.3}

% 章节格式
\titleformat{\section}
  {\normalfont\Large\bfseries\sffamily}{\thesection}{1em}{}
\titleformat{\subsection}
  {\normalfont\large\bfseries\sffamily}{\thesubsection}{1em}{}

% 列表格式
\setlist[itemize]{leftmargin=*, topsep=0pt}
\setlist[enumerate]{leftmargin=*, topsep=0pt}

% 定义笔记环境
\newenvironment{note}
  {\begin{framed}\small\itshape}
  {\end{framed}}

\newenvironment{important}
  {\begin{framed}\color{red}\small\bfseries}
  {\end{framed}}

% 代码环境
\usepackage{listings}
\lstset{
basicstyle=\ttfamily\small,
    breaklines=true,
    frame=single,
    numbers=left,
    numberstyle=\tiny\color{gray}
}

\begin{document}

% 标题
\begin{center}
    {\Huge\bfseries SSPU 高级程序语言设计笔记}\\[0.5em]
    {\large 2025年10月11日-2026年1月7日}\\[1em]
\end{center}

% 目录(可选)
\tableofcontents
\newpage


% 正文开始
\section{Python语言概述\textbf{(2025年10月15日)}}

% 这是2025年11月12日的笔记内容。python课程主要学习了以下内容:

% \begin{note}
%   Python中的函数调用是通过函数名加括号来实现的。例如,调用一个名为 \texttt{my\_function} 的函数可以写成 \texttt{my\_function()}。

% \end{note}

% \subsection{子节标题}

% 子节内容可以包含列表:

% \begin{itemize}
%   \item 要点一
%   \item 要点二
%   \item 要点三
% \end{itemize}

% 或者编号列表:

% \begin{enumerate}
%   \item 第一项
%   \item 第二项
%   \item 第三项
% \end{enumerate}

\newpage

\section{基本数据类型\textbf{(2025年10月22日)}}

% 数学公式示例:

% \begin{equation}
%   E = mc^2
% \end{equation}

% 或者行内公式:$a^2 + b^2 = c^2$

% \begin{important}
%   重要信息会用红色边框突出显示。
% \end{important}

\newpage

\section{程序控制结构\textbf{(2025年10月29日)}}

% \begin{lstlisting}[language=Python, caption=Python代码示例]
% def hello_world():
%     print("Hello, World!")
%     return True
% \end{lstlisting}

\newpage

\section{组合数据类型\textbf{(2025年11月5日)}}

\newpage

\section{函数与模块\textbf{(2025年11月12日)}}

\newpage

\end{document}